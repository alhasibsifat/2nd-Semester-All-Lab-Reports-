\documentclass[a4paper, 12pt]{article}%тип документа
\usepackage{osameet3}
%отступы
\usepackage[left=2cm,right=2cm,top=2cm,bottom=3cm,bindingoffset=0cm]{geometry}
\usepackage{wrapfig}
\usepackage{graphicx}
\graphicspath{{pictures/}}
\DeclareGraphicsExtensions{.pdf,.png,.jpg}
\usepackage{multirow}
\usepackage{pgfplots}
\pgfplotsset{compat=1.9}


%% provide authormark
\newcommand\authormark[1]{\textsuperscript{#1}}

%% standard packages and arguments should be modified as needed
\usepackage{amsmath, amsfonts, amssymb, amsthm, mathtools}
\usepackage[colorlinks=true,bookmarks=false,citecolor=blue,urlcolor=blue]{hyperref} %pdflatex
%\usepackage[breaklinks,colorlinks=true,bookmarks=false,citecolor=blue,urlcolor=blue]{hyperref} %latex w/dvipdf
\begin{document}

\title{Analysing the Interaction of Sodium Thiosulfate with Sulfuric Acid}

% \author{Author name(s)}
% \address{Author affiliation and full address}
% \email{e-mail address}
%%Uncomment the following line to override copyright year from the default current year.
%\copyrightyear{2022}

\author{Hasib Sifat,\authormark{1} Nikita Platonov,\authormark{2} and Kirill Alatortsev\authormark{3}}

\address{\authormark{1} Faculty of Physical and Quantum Electronics, 
\authormark{2,3}Faculty of Aerophysics and Space Research,  \authormark{1,2,3}Moscow Institute of Physics and Technology, Institutskiy Pereulok, 9, Dolgoprudny, Moscow Oblast, Russian Federation, 141701}




\begin{abstract}Back in 1889, Swedish chemist Svante Arrhenius proposed an equation based on the work of Dutch chemist Jacobus Henricus van 't Hoff who had noted in 1884 that the van 't Hoff equation for the temperature dependence of equilibrium constants suggests such a formula for the rates of both forward and reverse reactions. This equation has a vast and important application in determining rate of chemical reactions and for calculation of energy of activation. Arrhenius provided a physical justification and interpretation for the formula. It can be used to model the temperature variation of diffusion coefficients, population of crystal vacancies, creep rates, and many other thermally-induced processes/reactions. To see how the reaction rate has changes changing temperature, we did reactions of Sodium Thiosulfate with Sulfuric Acid at two different temperatures and observed the change in concentration with time. In this article, we have shown all the result we got from our experiment and conclude the applicability of the Van't-Hoff rule and the temperature dependence of all the components of Arrhenius Equation.
\end{abstract}

\begin{center}
    \section{Introduction}
\end{center}
The oxidative rusting of iron under Earth's atmosphere is a slow reaction that can take many years, but the combustion of cellulose in a fire is a reaction that takes place in fractions of a second. That means the term reaction rate can be defined by the speed at which a chemical reaction takes place and can be determined by measuring the changes in concentration over time. But it affected by a lot of factors, such as the nature of the reactant (ionic, covalent, homogeneous, heterogeneous), physical state of reactant ($gas>liquid>solid$), surface area of reactant, intensity of light, catalyst, temperature and concentration. Every factor has very important effect on the rate of reaction. Some factors can speed up the reaction depending on what reaction we're doing and some other factors can lower the speed. In our experiment, we independently kept our reactant and product from other factors and made reaction at room temperature first and then $10 \deg C$


\end{document}